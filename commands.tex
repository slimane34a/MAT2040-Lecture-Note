%%%%%%%%%%%%%%%%%%%
%%%%%%%%%%%%%%%%%%%%%%

%----------------------------------------------------------------------------------------
%	PACKAGES AND OTHER DOCUMENT CONFIGURATIONS
%----------------------------------------------------------------------------------------
\PassOptionsToPackage{table}{xcolor}
\documentclass[11pt,fleqn]{book} % Default font size and left-justified equations
\usepackage{graphics}
\usepackage{makeidx}
\makeindex
\usepackage[top=3cm,bottom=3cm,left=3.2cm,right=3.2cm,headsep=10pt,a4paper]{geometry} % Page margins
\usepackage{arydshln}
\usepackage{mathtools}
\usepackage{amsmath}
\usepackage{float}

\usepackage[all, pdf]{xy}
\usepackage{extarrows}
%\usepackage{xcolor} % Required for specifying colors by name
\usepackage[table]{xcolor}

\definecolor{ocre}{RGB}{243,102,25} % Define the orange color used for highlighting throughout the book
% Font Settings
\usepackage{avant} % Use the Avantgarde font for headings
%\usepackage{times} % Use the Times font for headings
\usepackage{mathptmx} % Use the Adobe Times Roman as the default text font together with math symbols from the Sym颅bol, Chancery and Com颅puter Modern fonts
\usepackage{microtype} % Slightly tweak font spacing for aesthetics
\usepackage[utf8]{inputenc} % Required for including letters with accents
\usepackage[T1]{fontenc} % Use 8-bit encoding that has 256 glyphs
\usepackage{bm}

\usepackage{fontenc}  
% Bibliography
\usepackage[style=alphabetic,sorting=nyt,sortcites=true,autopunct=true,babel=hyphen,hyperref=true,abbreviate=false,backref=true,backend=biber]{biblatex}
\addbibresource{bibliography.bib} % BibTeX bibliography file
\defbibheading{bibempty}{}
\usepackage{pgfplots}
\usepackage{subcaption}
% Index
\usepackage{calc} % For simpler calculation - used for 
\usepackage{makeidx} % Required to make an index
\makeindex % Tells LaTeX to create the files required for indexing
\usepackage{blkarray}
\usepackage{mathtools}
\DeclarePairedDelimiter\ceil{\lceil}{\rceil}
\DeclarePairedDelimiter\floor{\lfloor}{\rfloor}

%



%----------------------------------------------------------------------------------------









%----------------------------------------------------------------------------------------
%	VARIOUS REQUIlightblue PACKAGES
%----------------------------------------------------------------------------------------
\usepackage[table]{xcolor}
%\usepackage{titlesec} % Allows customization of titles
\usepackage{graphicx} % Requilightblue for including pictures
\graphicspath{{./Pictures/}} % Specifies the directory where pictures are stolightblue
\usepackage[bookmarks=true,colorlinks,linkcolor=black]{hyperref}
\usepackage{lipsum} % Inserts dummy text
\usepackage{mathdots}

\usepackage{tikz} % Requilightblue for drawing custom shapes

\usepackage[english]{babel} % English language/hyphenation

\usepackage{enumitem} % Customize lists
\setlist{nolistsep} % lightblueuce spacing between bullet points and numbelightblue lists

\usepackage{booktabs} % Requilightblue for nicer horizontal rules in tables

\usepackage{eso-pic} % Requilightblue for specifying an image background in the title page
\usepackage{colortbl}
\newcolumntype{U}{>{\columncolor[gray]{0.8}}c }
%----------------------------------------------------------------------------------------
%	MAIN TABLE OF CONTENTS
%----------------------------------------------------------------------------------------

\usepackage{titletoc} % Requilightblue for manipulating the table of contents\
\usepackage{tikz}

\usetikzlibrary{positioning}
\tikzset{>=stealth}
\newcommand{\tikzmark}[3][]
  {\tikz[remember picture, baseline]
    \node [anchor=base,#1](#2) {#3};}

\definecolor{lightblue}{cmyk}{1,0,0,0}
\contentsmargin{0cm} % Removes the default margin
% Chapter text styling
\titlecontents{chapter}[1.25cm] % Indentation
{\addvspace{15pt}\large\rmfamily\itshape\bfseries} % Spacing and font options for chapters
{\color{lightblue!60}\contentslabel[\Large\thecontentslabel]{1.25cm}\color{lightblue}} % Chapter number
{}  
{\color{lightblue!60}\large\itshape\rmfamily\bfseries\;\titlerule*[.5pc]{.}\;\thecontentspage} % Page number
% Section text styling
\titlecontents{section}[1.25cm] % Indentation
{\addvspace{5pt}\itshape\bfseries} % Spacing and font options for sections
{\contentslabel[\thecontentslabel]{1.25cm}} % Section number
{}
{\rmfamily\itshape\hfill\color{black}\thecontentspage} % Page number
[]
% Subsection text styling
\titlecontents{subsection}[1.25cm] % Indentation
{\addvspace{1pt}\rmfamily\itshape\mdseries\small} % Spacing and font options for subsections
{\contentslabel[\thecontentslabel]{1.25cm}} % Subsection number
{}
{\rmfamily\itshape\;\titlerule*[.5pc]{.}\;\thecontentspage} % Page number
[] 

%----------------------------------------------------------------------------------------
%	MINI TABLE OF CONTENTS IN CHAPTER HEADS
%----------------------------------------------------------------------------------------

% Section text styling
\titlecontents{lsection}[0em] % Indendating
{\footnotesize\rmfamily} % Font settings
{}
{}
{}

% Subsection text styling
\titlecontents{lsubsection}[.5em] % Indentation
{\normalfont\footnotesize\rmfamily} % Font settings
{}
{}
{}
 
%----------------------------------------------------------------------------------------
%	PAGE HEADERS
%----------------------------------------------------------------------------------------

\usepackage{fancyhdr} % Requilightblue for header and footer configuration

\pagestyle{fancy}
\renewcommand{\chaptermark}[1]{\markboth{\rmfamily\normalsize\bfseries #1}{}} % Chapter text font settings
\renewcommand{\sectionmark}[1]{\markright{\bfseries\itshape\normalsize\thesection\hspace{5pt}#1}{}} % Section text font settings
\fancyhf{} \fancyhead[LE,RO]{\bfseries\normalsize\thepage} % Font setting for the page number in the header
\fancyhead[LO]{\rightmark} % Print the nearest section name on the left side of odd pages
\fancyhead[RE]{\leftmark} % Print the current chapter name on the right side of even pages
\renewcommand{\headrulewidth}{0.5pt} % Width of the rule under the header
\addtolength{\headheight}{2.5pt} % Increase the spacing around the header slightly
\renewcommand{\footrulewidth}{0pt} % Removes the rule in the footer
\fancypagestyle{plain}{\fancyhead{}\renewcommand{\headrulewidth}{0pt}} % Style for when a plain pagestyle is specified

% Removes the header from odd empty pages at the end of chapters
\makeatletter
\renewcommand{\cleardoublepage}{
\clearpage\ifodd\c@page\else
\hbox{}
\vspace*{\fill}
\thispagestyle{empty}
\newpage
\fi}

%----------------------------------------------------------------------------------------
%	THEOREM STYLES
%----------------------------------------------------------------------------------------

\usepackage{amsmath,amsfonts,amssymb,amsthm} % For including math equations, theorems, symbols, etc

\newcommand{\intoo}[2]{\mathopen{]}#1\,;#2\mathclose{[}}
\newcommand{\ud}{\mathop{\mathrm{{}d}}\mathopen{}}
\newcommand{\intff}[2]{\mathopen{[}#1\,;#2\mathclose{]}}
\newcommand\x{\times}
\newcommand\bigzero{\makebox(0,0){\text{\huge0}}}
\newcommand*{\bord}{\multicolumn{1}{c|}{}}
\newtheorem{notation}{Notation}[chapter]
\renewcommand{\emph}{\textbf}

\newtheoremstyle{lightbluenum} % Theorem style name
{7pt} % Space above
{7pt} % Space below
{\normalfont} % Body font
{} % Indent amount
{\small\bf\rmfamily\color{lightblue}} % Theorem head font
{\;\;} % Punctuation after theorem head
{0.25em} % Space after theorem head
{\small\rmfamily\color{lightblue}\thmname{#1}\thmnumber{\@ifnotempty{#1}{ }\@upn{#2}} % Theorem text (e.g. Theorem 2.1)
\thmnote{\ {\the\thm@notefont\rmfamily\bfseries\color{black}--- #3.}}} % Optional theorem note
\renewcommand{\qedsymbol}{$\blacksquare$} % Optional qed square

\newtheoremstyle{blacknumex} % Theorem style name
{7pt} % Space above
{7pt} % Space below
{\normalfont} % Body font
{} % Indent amount
{\small\bf\rmfamily} % Theorem head font
{\;\;} % Punctuation after theorem head
{0.25em} % Space after theorem head
{\small\rmfamily{\tiny\ensuremath{\blacksquare}}\ \thmname{#1}\thmnumber{\@ifnotempty{#1}{ }\@upn{#2}} % Theorem text (e.g. Theorem 2.1)
\thmnote{\ {\the\thm@notefont\rmfamily\bfseries--- #3.}}} % Optional theorem note

\newtheoremstyle{blacknum} % Theorem style name
{7pt} % Space above
{7pt} % Space below
{\normalfont} % Body font
{} % Indent amount
{\small\bf\rmfamily} % Theorem head font
{\;\;} % Punctuation after theorem head
{0.25em} % Space after theorem head
{\small\rmfamily\thmname{#1}\thmnumber{\@ifnotempty{#1}{ }\@upn{#2}} % Theorem text (e.g. Theorem 2.1)
\thmnote{\ {\the\thm@notefont\rmfamily\bfseries--- #3.}}} % Optional theorem note
\makeatother

% Defines the theorem text style for each type of theorem to one of the three styles above
\theoremstyle{lightbluenum}
\newtheorem{theoremeT}{Theorem}[chapter]
\newtheorem{proposition}{Proposition}[chapter]
\newtheorem{problem}{Problem}[chapter]
\newtheorem{exerciseT}{Exercise}[chapter]
\theoremstyle{blacknumex}
\newtheorem{exampleT}{Example}[chapter]
\newtheorem{SolutionT}{Solution}[chapter]
\theoremstyle{blacknum}
\newtheorem{vocabulary}{Vocabulary}[chapter]
\newtheorem{definitionT}{Definition}[chapter]
\newtheorem{corollaryT}{Corollary}[chapter]
\newcommand{\trans}{^{\mathrm T}}
\newcommand{\Her}{^{\mathrm H}}
%----------------------------------------------------------------------------------------
%	DEFINITION OF COLOlightblue BOXES
%----------------------------------------------------------------------------------------

\RequirePackage[framemethod=default]{mdframed} % Requilightblue for creating the theorem, definition, exercise and corollary boxes

% Theorem box
\newmdenv[skipabove=7pt,
skipbelow=7pt,
backgroundcolor=black!5,
linecolor=lightblue,
innerleftmargin=5pt,
innerrightmargin=5pt,
innertopmargin=5pt,
leftmargin=0cm,
rightmargin=0cm,
innerbottommargin=5pt]{tBox}

% Exercise box	  
\newmdenv[skipabove=7pt,
skipbelow=7pt,
rightline=false,
leftline=true,
topline=false,
bottomline=false,
backgroundcolor=lightblue!10,
linecolor=lightblue,
innerleftmargin=5pt,
innerrightmargin=5pt,
innertopmargin=5pt,
innerbottommargin=5pt,
leftmargin=0cm,
rightmargin=0cm,
linewidth=4pt]{eBox}	

% Definition box
\newmdenv[skipabove=10pt,
skipbelow=10pt,
rightline=false,
leftline=true,
topline=false,
bottomline=false,
linecolor=lightblue,
innerleftmargin=5pt,
innerrightmargin=5pt,
innertopmargin=0pt,
leftmargin=0cm,
rightmargin=0cm,
linewidth=4pt,
innerbottommargin=0pt]{dBox}	

% Corollary box
\newmdenv[skipabove=7pt,
skipbelow=7pt,
rightline=false,
leftline=true,
topline=false,
bottomline=false,
linecolor=gray,
backgroundcolor=black!5,
innerleftmargin=5pt,
innerrightmargin=5pt,
innertopmargin=5pt,
leftmargin=0cm,
rightmargin=0cm,
linewidth=4pt,
innerbottommargin=5pt]{cBox}				  
		  

% Creates an environment for each type of theorem and assigns it a theorem text style from the "Theorem Styles" section above and a cololightblue box from above
\newenvironment{theorem}{\begin{tBox}\begin{theoremeT}}{\end{theoremeT}\end{tBox}}
\newenvironment{exercise}{\begin{eBox}\begin{exerciseT}}{\hfill{\color{lightblue}\tiny\ensuremath{\blacksquare}}\end{exerciseT}\end{eBox}}				 
%\newenvironment{defintion}{\begin{eBox}\begin{definitionT}}{\hfill{\color{lightblue}\tiny\ensuremath{\blacksquare}}\end{definitionT}\end{eBox}}
\newenvironment{definition}{\begin{eBox}\begin{definitionT}}{\hfill{\color{lightblue}\tiny\ensuremath{\blacksquare}}\end{definitionT}\end{eBox}}
\newenvironment{example}{\begin{eBox}\begin{exampleT}}{\hfill{\color{lightblue}\tiny\ensuremath{\blacksquare}}\end{exampleT}\end{eBox}}
\newenvironment{Solution}{\begin{eBox}\begin{SolutionT}}{\hfill{\color{lightblue}\tiny\ensuremath{\blacksquare}}\end{SolutionT}\end{eBox}}
%\newenvironment{example}{\begin{exampleT}}{\hfill{\tiny\ensuremath{\blacksquare}}\end{exampleT}}		
\newenvironment{corollary}{\begin{cBox}\begin{corollaryT}}{\end{corollaryT}\end{cBox}}	

%----------------------------------------------------------------------------------------
%	REMARK ENVIRONMENT
%----------------------------------------------------------------------------------------

\newenvironment{remark}{\par\vskip10pt\small % Vertical white space above the remark and smaller font size
\begin{list}{}{
\leftmargin=35pt % Indentation on the left
\rightmargin=25pt}\item\ignorespaces % Indentation on the right
\makebox[-2.5pt]{\begin{tikzpicture}[overlay]
\node[draw=lightblue!60,line width=1pt,circle,fill=lightblue!25,font=\rmfamily\bfseries,inner sep=2pt,outer sep=0pt] at (-15pt,0pt){\textcolor{lightblue}{R}};\end{tikzpicture}} % Orange R in a circle
\advance\baselineskip -1pt}{\end{list}\vskip5pt} % Tighter line spacing and white space after remark

%----------------------------------------------------------------------------------------
%	SECTION NUMBERING IN THE MARGIN
%----------------------------------------------------------------------------------------

\makeatletter
{\normalfont\LARGE\rmfamily\bfseries}
\renewcommand{\@seccntformat}[1]{\llap{\textcolor{lightblue}{\csname the#1\endcsname}\hspace{1em}}}                    
\renewcommand{\section}{\@startsection{section}{1}{\z@}
{-4ex \@plus -1ex \@minus -.4ex}
{1ex \@plus.2ex }
{\normalfont\huge\rmfamily\bfseries}}
\renewcommand{\subsection}{\@startsection {subsection}{2}{\z@}
{-3ex \@plus -0.1ex \@minus -.4ex}
{0.5ex \@plus.2ex }
{\normalfont\LARGE\itshape\bfseries}}
\renewcommand{\subsubsection}{\@startsection {subsubsection}{3}{\z@}
{-2ex \@plus -0.1ex \@minus -.2ex}
{0.2ex \@plus.2ex }
{\normalfont\normalsize\rmfamily\bfseries\itshape}}                        
\renewcommand\paragraph{\@startsection{paragraph}{4}{\z@}
{-2ex \@plus-.2ex \@minus .2ex}
{0.1ex}
{\normalfont\small\rmfamily\bfseries}}

%----------------------------------------------------------------------------------------
%	CHAPTER HEADINGS
%----------------------------------------------------------------------------------------

\newcommand{\thechapterimage}{}
\newcommand{\chapterimage}[1]{\renewcommand{\thechapterimage}{#1}}
\def\thechapter{\arabic{chapter}}
\def\@makechapterhead#1{
\thispagestyle{empty}
{\centering \normalfont\rmfamily
\ifnum \c@secnumdepth >\m@ne
\if@mainmatter
\startcontents
\begin{tikzpicture}[remember picture,overlay]
\node at (current page.north west)
{\begin{tikzpicture}[remember picture,overlay]

\node[anchor=north west] at (-4pt,4pt) {\includegraphics[width=\paperwidth]{\thechapterimage}};

%Commenting the 3 lines below removes the small contents box in the chapter heading
%\draw[fill=white,opacity=.6] (1cm,0) rectangle (8cm,-7cm);
%\node[anchor=north west] at (1cm,.25cm) {\parbox[t][8cm][t]{6.5cm}{\huge\bfseries\flushleft \printcontents{2}{2}{\setcounter{tocdepth}{2}}}};

\draw[anchor=west] (5cm,-9cm) node [rounded corners=25pt,fill=white,opacity=.7,inner sep=15.5pt]{\huge\rmfamily\bfseries\textcolor{black}{\vphantom{plPQq}\makebox[22cm]{}}};
\draw[anchor=west] (5cm,-9cm) node [rounded corners=25pt,draw=lightblue,line width=2pt,inner sep=15pt]{\huge\rmfamily\itshape\bfseries\textcolor{black}{\thechapter\ ---\ #1\vphantom{plPQq}\makebox[22cm]{}}};
\end{tikzpicture}};
\end{tikzpicture}}\par\vspace*{230\p@}
}
\def\@makeschapterhead#1{
\thispagestyle{empty}
{\centering \normalfont\rmfamily
\ifnum \c@secnumdepth >\m@ne
\if@mainmatter
\startcontents
\begin{tikzpicture}[remember picture,overlay]
\node at (current page.north west)
{\begin{tikzpicture}[remember picture,overlay]
\node[anchor=north west] at (-4pt,4pt) {\includegraphics[width=\paperwidth]{\thechapterimage}};
\draw[anchor=west] (5cm,-9cm) node [rounded corners=25pt,fill=white,opacity=.7,inner sep=15.5pt]{\huge\rmfamily\bfseries\textcolor{black}{\vphantom{plPQq}\makebox[22cm]{}}};
\draw[anchor=west] (5cm,-9cm) node [rounded corners=25pt,draw=lightblue,line width=2pt,inner sep=15pt]{\huge\rmfamily\bfseries\textcolor{black}{#1\vphantom{plPQq}\makebox[22cm]{}}};
\end{tikzpicture}};
\end{tikzpicture}}\par\vspace*{230\p@}
}
\makeatother
\newcommand{\diff}{\,\mathrm{d}}
\DeclareMathOperator{\rank}{rank}
\DeclareMathOperator{\Span}{span}
\DeclareMathOperator{\row}{row}
\DeclareMathOperator{\col}{col}
\DeclareMathOperator{\Range}{Range}
\DeclareMathOperator{\im}{Im}
\DeclarePairedDelimiterX{\inp}[2]{\langle}{\rangle}{#1, #2}
\DeclareMathOperator{\Proj}{Proj}
\newcommand{\degree}{^\circ}
\DeclareMathOperator{\diag}{diag}
\newcommand*\conj[1]{\bar{#1}}
\DeclareMathOperator{\eig}{eig}
\DeclareMathOperator{\trace}{trace}
\DeclareMathOperator{\range}{range}
\DeclareMathOperator{\sign}{sign}
\DeclareMathOperator{\tr}{tr}